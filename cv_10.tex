%%%%%%%%%%%%%%%%%%%%%%%%%%%%%%%%%%%%%%%%%
% Friggeri Resume/CV
% XeLaTeX Template
% Version 1.0 (5/5/13)
%
% This template has been downloaded from:
% http://www.LaTeXTemplates.com
%
% Original author:
% Adrien Friggeri (adrien@friggeri.net)
% https://github.com/afriggeri/CV
%
% License:
% CC BY-NC-SA 3.0 (http://creativecommons.org/licenses/by-nc-sa/3.0/)
%
% Important notes:
% This template needs to be compiled with XeLaTeX and the bibliography, if used,
% needs to be compiled with biber rather than bibtex.
%
%%%%%%%%%%%%%%%%%%%%%%%%%%%%%%%%%%%%%%%%%

\documentclass[a4paper]{friggeri-cv} % Add 'print' as an option into the square bracket to remove colors from this template for printing

\begin{document}

\header{Pierre-Louis}{TILAK}{Embedded System Engineer} % Your name and current job title/field

%----------------------------------------------------------------------------------------
%	SIDEBAR SECTION
%----------------------------------------------------------------------------------------

\begin{aside} % In the aside, each new line forces a line break
\section{Contact}
115 Rue Bonnat
Appartement 244
31400 Toulouse
France
06 17 83 77 13
~
\href{mailto:pierre.tilak@gmail.com}{pierre.tilak@gmail.com}
\href{https://www.linkedin.com/pub/pierre-louis-tilak/96/162/a83}{linkedin://pierre-louis-tilak}
\section{Languages}
French : Mothertongue
English : Fluent
TOEIC : 925 pts 
Spanish : Academic Level\bigskip\bigskip
\section{Programming Skills}
%{\color{red} $\varheartsuit$} C/Cpp
C, CPP
Python, 
Docker, YML,
HTML,\bigskip\bigskip
\section{Hardware \& Computer}
Embedded Linux\bigskip\bigskip
\section{Software}
Vim, Git/GitLab, Docker, \href{https://github.com/tilaktilak/CV}{Latex}, MS Office\bigskip\bigskip
\section{Other}
Project Management : Agile, Scrum	\bigskip\bigskip
\section{Interest}
\textbf{RC Models} Design and build RC Gliders and Airplanes
\textbf{Music} Guitar
\textbf{Sport} Paragliding, Climbing, Ski, MountainBike
\end{aside}

%----------------------------------------------------------------------------------------
%	EDUCATION SECTION
%----------------------------------------------------------------------------------------

\section{Education}

\begin{entrylist}

%------------------------------------------------
\entry
{2010--2014}
{Master of engineering}
{Institut National des Sciences Appliquées, Toulouse}
{Automatic Control and Electronics
\begin{itemize}
\item Real-time control 
\item Electronic circuit design
\item Low Level Programming : Drivers, STM32
\item Critical Embedded System Programming : SCADE
\item Real Time OS, JAVA, Android
\end{itemize}}

%------------------------------------------------

\entry
{2013}
{Exchange Semester \textit{Thailand}}
{Chulanlongkorn University, Bangkok}
{Information and Communication Engineering : 
Universal and Technical English skill refining, Embedded System, Arduino Development, Multimedia Engineering}

%------------------------------------------------

%\entry
%{2010}
%{Baccalauréat Scientifique Option Sciences de l'Ingénieur}
%{Déodat de Séverac, Toulouse}{Mention Bien}
%------------------------------------------------

\end{entrylist}

%----------------------------------------------------------------------------------------
%	WORK EXPERIENCE SECTION
%----------------------------------------------------------------------------------------

\section{Experiences}

\begin{entrylist}
%------------------------------------------------
\entry
{2015-2019}
{Delair \textit{Developper in Autopilot team}}
{Toulouse}
{Embedded System Engineer
\begin{itemize}
\item C Programming on microcontroller : Sensor interface, Drivers, Communication protocols.
\item Control Laws for Brushelss Gimbal : Kalman filtering, Brushless model control, FOC control
\item Python engineering tools : GUI for experiment and log analysis ( PyQT, Matplotlib, Dash \& Plotly ), Scripts to communicate (TCP, Serial Port), Web request, Unitary Tests and Continuious integration (Behave)
\item Implementation of Developpement Environnement : Cross-compilation toolchain (arm-gcc, make), Continious Integration with GitLab Runners : Create docker images (build, complete test environement with simulator) Automatic release package generation and versionning (Gitlab Pipeline, JFrog Artifactory)
\item CPP Programming on embedded linux : Drone sensor configuration, Thread and IPC (  Linux Signals, SharedMemory, ZMQ, ...)
\end{itemize}
%
}
%------------------------------------------------
\entry
{2014}
{SCLE SFE \textit{3 months internship}}
{Toulouse}
{Test plan development : JTAG Boundary Scan (XJTAG Software)
\begin{itemize}
\item Test code for SoC (Ethernet, SPI, I2C, Serial, Watchdog ...)
\item Design and make interface boards for test bench
\end{itemize}
}
%{Développement de plan de test JTAG pour Boundary Scan (Logiciel XJTAG):\smallskip
%\begin{itemize}
%\item Codes test pour les SoC (composants Ethernet, SPI, I2C, Watchdog, Série)
%\item Analyse, Conception et Réalisation de cartes interfaces, débogage sur banc de test.
%\end{itemize}}

%------------------------------------------------
\entry
{2012}
{Polymont Subcontractor of Airbus \textit{Operating Method Writer}}
{Toulouse}
{ Writing operating method for maintenance on Iron-bird Airbus A330, A320 and A380 }
%{Mission sur démantèlement de site : Rédaction de modes opératoires sur Iron-bird Airbus A330,A320 et A380.}
%\entry
%{2012}
%{Rédacteur de modes opératoires}
%{Polymont Sous-traitant Airbus - Toulouse}
%{Mission sur démantèlement de site : Rédaction de modes opératoires sur Iron-bird Airbus A330,A320 et A380.}

%------------------------------------------------

%\entry{2011}{Stage Ouvrier 1 mois}{Airbus - Toulouse}{Equipe Inspection Qualité A380 : Optimisation de la durée des inspections}


\end{entrylist}


%----------------------------------------------------------------------------------------
%	PROJETS
%----------------------------------------------------------------------------------------

\section{Personal Projects}

\begin{entrylist}
%------------------------------------------------

\entry
{2016}
{OpenSource Cocktail Machine}
{  }
{Design and build a cocktail machine (Team of three engineers)
\begin{itemize}
\item \href{https://cad.onshape.com/documents/fb6038b2912b9dee202ae483/w/fca3360b8bc95b6a2f76da72/e/469430195367acaf31facc33}{CAD Model with Onshape.com}
\item { Hardware development : Electronic board to interface Raspberry/ESP8266 and stepper motors }
\item {\href{https://github.com/tilaktilak/Inebriator}{Software development} : \
Python webserver and lowlevel code for Raspberry, Lua webserver and low level interface for ESP8266}
\item {HTML, CSS, CherryPy, NodeMCU framework on ESP8266}
%\item { CHallenges : HTML + CSS webinterface, EMC with high current stepper motors, Memory usage over EPS8266 (lua compilation), Handle multiple request from web interface, .. }
\end{itemize}
}

\entry
{2018}
{OpenSource Paragliding Variometer GPS and SD logger (IGC)}
{}
{Design and build a paragliding variometer GPS
\begin{itemize}
\item Autonomous Variometer (\href{https://github.com/tilaktilak/variometre}{Source Code} - \href{https://cad.onshape.com/documents/a2edef2565390c2a63887be6/w/e115fc79a19f6ea7701f931d/e/4bf08f905543160a6d4a78ce}{CAD Model} 
and Interface with a Kobo eReader \href{https://github.com/tilaktilak/xcvario}{Source Code} - \href{https://cad.onshape.com/documents/e35f287762e7eb2639163738/w/944245d017b66dfac14ea554/e/e65c9aaedd3effabe6219e48}{CAD Model})
%\item Design and build the electronic board (Atmega328p, Ublox Neo6M, BME280, ...)
%\item Develop code based on Arduino
\item Kalman Filtering, Optimise memory and time execution, CAD modeling, 3D printing
\end{itemize}}

%\entry
%{6 Mois}
%{AR-Drone et jeu de plateau}
%{Projet 5A - INSA, Toulouse}
%{AR-Drone Parrot, Cortex A8, Linux, OpenCV \smallskip
%\begin{itemize}
%\item Développement des primitives pour faire évoluer le drone
%\item Détection de l'environnement via traitement de l'image
%\item IHM interactive et contrôle du drone via superviseur
%\end{itemize}
%}
%
%%------------------------------------------------
%
%%------------------------------------------------
%
%\entry
%{6 Mois}
%{Développement d'un OS de Console de Jeu}
%{Projet Tutoré 4A - INSA, Toulouse}
%{Carte de Développement Keil STM32\smallskip
%\begin{itemize}
%\item Développement des drivers (LCD,FSMC,SDIO,etc), codes d'initialisation du matériel
%\item Notions de contexte applicatif et de processus
%\item Chargement des informations en RAM et exécution d'un jeu depuis l'OS
%\end{itemize}
%}
%
%%------------------------------------------------
%
%%\entry
%%{2013}
%%{Projet Électronique}
%%{Chulalongkorn, Bangkok}
%%{1 mois Carte Arduino, Développement d'une boîte à musique,
%%Codage d'un ensemble de primitives permettant de choisir, lancer et arrêter la lecture d'une piste audio}
%%\end{entrylist}
%
%%------------------------------------------------
%
%
%\entry
%{2014}
%{Conception et Développement Temps Réel}
%{Projet Temps Réel 4A - INSA, Toulouse}
%{Conception UML, Définition des threads, des primitives de synchronisation, développement sur OS Xenomai afin de faire évoluer un robot filmé dans une arène.}
%
%\entry
%{2013}
%{Assembleur ARM Cortex M3}
%{Projet STM32 4A - INSA, Toulouse}
%{Programmation bas niveau, données accéléromètres, afficheur à persistance rétinienne}
\end{entrylist}

%----------------------------------------------------------------------------------------
%	COMMUNICATION SKILLS SECTION
%----------------------------------------------------------------------------------------

%\section{compétences}
%
%%------------------------------------------------
%
%\emph{Systèmes Embarqués}
%
%\begin{itemize}
%\item{\textbf{Programmation }}{ Langage C, Java, Assembleur, SQL, ADA, HTML, Xjease}
%\item{\textbf{Projets }}{STM32 Cortex M3, Pics, Arduino}
%\item{\textbf{Hardware PC }}{Bonnes notions sur l'architecture, Développement bas niveau (Assembleur, Drivers,etc)}
%\item{\textbf{OS utilisés }}{Linux, Xenomai (OS Temps Réel), Windows}
%\item{\textbf{Logiciels Maîtrisés }}{Intelij-Idea, Matlab-Simulink, Keil-uVision, Cadence Orcad, Solidwork} 
%\item{\textbf{Bureautique }}{SubVersion, Tortoise, SVN, Latex, Pack Office, Gestion de base de données }
%\end{itemize}
%
%%------------------------------------------------
%
%\emph{Autres}
%\begin{itemize}
%\item{\textbf{Gestion de Projet } Mis en pratique lors de multiples projets INSA ainsi que lors de mes stages}
%\item{ \textbf{Oral } Attrait pour les présentations orales, le partage des informations}
%\item{ \textbf{Gestion \& Comptabilité } Notions pratiques acquises lors de cours et de projets de simulation d'entreprise}
%\end{itemize}
%
%%------------------------------------------------



%----------------------------------------------------------------------------------------
%	INTERESTS SECTION
%----------------------------------------------------------------------------------------

%\section{Hobbies}
%\textbf{Modélisme} Conception et fabrication d'avions radio-commandés
%\textbf{Électronique et Informatique} Programmation de pic, Électronique, Développement Linux
%\textbf{Musique} Guitariste amateur
%\textbf{Sport} Paragliding, VTT, Basket-ball, Ski

%\section{Interests}

\end{document}
