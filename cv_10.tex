%%%%%%%%%%%%%%%%%%%%%%%%%%%%%%%%%%%%%%%%%
% Friggeri Resume/CV
% XeLaTeX Template
% Version 1.0 (5/5/13)
%
% This template has been downloaded from:
% http://www.LaTeXTemplates.com
%
% Original author:
% Adrien Friggeri (adrien@friggeri.net)
% https://github.com/afriggeri/CV
%
% License:
% CC BY-NC-SA 3.0 (http://creativecommons.org/licenses/by-nc-sa/3.0/)
%
% Important notes:
% This template needs to be compiled with XeLaTeX and the bibliography, if used,
% needs to be compiled with biber rather than bibtex.
%
%%%%%%%%%%%%%%%%%%%%%%%%%%%%%%%%%%%%%%%%%

\documentclass[a4paper]{friggeri-cv} % Add 'print' as an option into the square bracket to remove colors from this template for printing

%\addbibresource{bibliography.bib} % Specify the bibliography file to include publications

\begin{document}

\header{Pierre-Louis}{TILAK}{Embedded System Engineer} % Your name and current job title/field
%\header{Master Droit }{Juriste d'Entreprise}{Maud BERJAUD-ZARZI}{ID.jpg} % Your name and current job title/field

%----------------------------------------------------------------------------------------
%	SIDEBAR SECTION
%----------------------------------------------------------------------------------------

\begin{aside} % In the aside, each new line forces a line break
\section{Contact}
115 Rue Bonnat apt 244
31400 Toulouse
France
06 17 83 77 13
~
\href{mailto:pierre.tilak@gmail.com}{pierre.tilak@gmail.com}
%\href{etud.insa-toulouse.fr/~tilak}{etud.insa-toulouse.fr/~tilak}
\href{https://www.linkedin.com/pub/pierre-louis-tilak/96/162/a83}{linkedin://pierre-louis-tilak}
\section{Languages}
English : Fluent
TOEIC : 925 pts 
Spanish : Academic Level\bigskip\bigskip
\section{Programming Skills}
%{\color{red} $\varheartsuit$} C/Cpp
C, CPP
{\color{red} $\varheartsuit$} Python, 
Docker, YML,
Javascript,
HTML,\bigskip\bigskip
\section{Hardware \& Computer}
Embedded Linux\bigskip\bigskip
\section{Software}
Vim, Git/GitLab, Docker, Latex, MS Office\bigskip\bigskip
\section{Other}
Project Management : Agile, Scrum	
\end{aside}

%----------------------------------------------------------------------------------------
%	EDUCATION SECTION
%----------------------------------------------------------------------------------------

\section{Education}

\begin{entrylist}

%------------------------------------------------
\entry
{2010--2014}
{Master of engineering}
{Institut National des Sciences Appliquées, Toulouse}
{ Automatic Control and Electronics
\begin{itemize}
\item Real-time control 
\item Electronic circuit design
\item Low Level Programming : Drivers, STM32
\item Critical Embedded System Programming : SCADE
\item Real Time OS, JAVA, Android
\end{itemize}}

%------------------------------------------------

\entry
{2013}
{Exchange Semester - Thailand}
{Study at Chulanlongkorn University - Bangkok}
{Information and Communication Engineering : 
Universal and Technical English skill refining, Embedded System, Arduino Development, Multimedia Engineering}

%------------------------------------------------

%\entry
%{2010}
%{Baccalauréat Scientifique Option Sciences de l'Ingénieur}
%{Déodat de Séverac, Toulouse}{Mention Bien}
%------------------------------------------------



\end{entrylist}




%----------------------------------------------------------------------------------------
%	WORK EXPERIENCE SECTION
%----------------------------------------------------------------------------------------

\section{Experiences}

\begin{entrylist}
%------------------------------------------------
\entry
{Now}
{Delair}
{Work as developper in Autopilot team}
{Embedded System Engineer
\begin{itemize}
\item C Programming on microcontroller : Control laws for a brushless Gimbal, Drivers, Communication protocols.
\item Python engineering tools : GUI for experiment and log analysis ( PyQT, Matplotlib, Dash \& Plotly ), Scripts to communicate (TCP, Serial Port), Web request, Unitary Tests and Continuious integration (Behave)
\item Implementation of Developpement Environnement : Cross-compilation toolchain (arm-gcc, make), Continious Integration with GitLab Runners : Create docker images (build, complete test environement with simulator) Automatic release package generation and versionning (Gitlab Pipeline, JFrog Artifactory)
\item CPP Programming on embedded linux : Drone sensor configuration, Thread and IPC (  Linux Signals, SharedMemory, ZMQ, ...)
\end{itemize}
%
}
%------------------------------------------------
\entry
{2014}
{Stage Ingénieur 3 mois}
{SCLE SFE - Toulouse}
{Développement de plan de test JTAG pour Boundary Scan (Logiciel XJTAG):\smallskip
\begin{itemize}
\item Codes test pour les SoC (composants Ethernet, SPI, I2C, Watchdog, Série)
\item Analyse, Conception et Réalisation de cartes interfaces, débogage sur banc de test.
\end{itemize}}

%------------------------------------------------

\entry
{2012}
{Rédacteur de modes opératoires}
{Polymont Sous-traitant Airbus - Toulouse}
{Mission sur démantèlement de site : Rédaction de modes opératoires sur Iron-bird Airbus A330,A320 et A380.}

%------------------------------------------------

\entry{2011}{Stage Ouvrier 1 mois}{Airbus - Toulouse}{Equipe Inspection Qualité A380 : Optimisation de la durée des inspections}


\end{entrylist}


%----------------------------------------------------------------------------------------
%	PROJETS
%----------------------------------------------------------------------------------------

\section{projets tutorés}

\begin{entrylist}
%------------------------------------------------

\entry
{6 Mois}
{AR-Drone et jeu de plateau}
{Projet 5A - INSA, Toulouse}
{AR-Drone Parrot, Cortex A8, Linux, OpenCV \smallskip
\begin{itemize}
\item Développement des primitives pour faire évoluer le drone
\item Détection de l'environnement via traitement de l'image
\item IHM interactive et contrôle du drone via superviseur
\end{itemize}
}

%------------------------------------------------

%------------------------------------------------

\entry
{6 Mois}
{Développement d'un OS de Console de Jeu}
{Projet Tutoré 4A - INSA, Toulouse}
{Carte de Développement Keil STM32\smallskip
\begin{itemize}
\item Développement des drivers (LCD,FSMC,SDIO,etc), codes d'initialisation du matériel
\item Notions de contexte applicatif et de processus
\item Chargement des informations en RAM et exécution d'un jeu depuis l'OS
\end{itemize}
}

%------------------------------------------------

%\entry
%{2013}
%{Projet Électronique}
%{Chulalongkorn, Bangkok}
%{1 mois Carte Arduino, Développement d'une boîte à musique,
%Codage d'un ensemble de primitives permettant de choisir, lancer et arrêter la lecture d'une piste audio}
%\end{entrylist}

%------------------------------------------------


\entry
{2014}
{Conception et Développement Temps Réel}
{Projet Temps Réel 4A - INSA, Toulouse}
{Conception UML, Définition des threads, des primitives de synchronisation, développement sur OS Xenomai afin de faire évoluer un robot filmé dans une arène.}

\entry
{2013}
{Assembleur ARM Cortex M3}
{Projet STM32 4A - INSA, Toulouse}
{Programmation bas niveau, données accéléromètres, afficheur à persistance rétinienne}
\end{entrylist}

%----------------------------------------------------------------------------------------
%	COMMUNICATION SKILLS SECTION
%----------------------------------------------------------------------------------------

%\section{compétences}
%
%%------------------------------------------------
%
%\emph{Systèmes Embarqués}
%
%\begin{itemize}
%\item{\textbf{Programmation }}{ Langage C, Java, Assembleur, SQL, ADA, HTML, Xjease}
%\item{\textbf{Projets }}{STM32 Cortex M3, Pics, Arduino}
%\item{\textbf{Hardware PC }}{Bonnes notions sur l'architecture, Développement bas niveau (Assembleur, Drivers,etc)}
%\item{\textbf{OS utilisés }}{Linux, Xenomai (OS Temps Réel), Windows}
%\item{\textbf{Logiciels Maîtrisés }}{Intelij-Idea, Matlab-Simulink, Keil-uVision, Cadence Orcad, Solidwork} 
%\item{\textbf{Bureautique }}{SubVersion, Tortoise, SVN, Latex, Pack Office, Gestion de base de données }
%\end{itemize}
%
%%------------------------------------------------
%
%\emph{Autres}
%\begin{itemize}
%\item{\textbf{Gestion de Projet } Mis en pratique lors de multiples projets INSA ainsi que lors de mes stages}
%\item{ \textbf{Oral } Attrait pour les présentations orales, le partage des informations}
%\item{ \textbf{Gestion \& Comptabilité } Notions pratiques acquises lors de cours et de projets de simulation d'entreprise}
%\end{itemize}
%
%%------------------------------------------------



%----------------------------------------------------------------------------------------
%	INTERESTS SECTION
%----------------------------------------------------------------------------------------

\section{Hobbies}
\textbf{Modélisme} Conception et fabrication d'avions radio-commandés
\textbf{Électronique et Informatique} Programmation de pic, Électronique, Développement Linux
\textbf{Musique} Guitariste amateur
\textbf{Sport} Paragliding, VTT, Basket-ball, Ski



%----------------------------------------------------------------------------------------
%	PUBLICATIONS SECTION
%----------------------------------------------------------------------------------------

%\section{publications}
%
%\printbibsection{article}{article in peer-reviewed journal} % Print all articles from the bibliography
%
%\printbibsection{book}{books} % Print all books from the bibliography
%
%\begin{refsection} % This is a custom heading for those references marked as "inproceedings" but not containing "keyword=france"
%\nocite{*}
%\printbibliography[sorting=chronological, type=inproceedings, title={international peer-reviewed conferences/proceedings}, notkeyword={france}, heading=subbibliography]
%\end{refsection}
%
%\begin{refsection} % This is a custom heading for those references marked as "inproceedings" and containing "keyword=france"
%\nocite{*}
%\printbibliography[sorting=chronological, type=inproceedings, title={local peer-reviewed conferences/proceedings}, keyword={france}, heading=subbibliography]
%\end{refsection}
%
%\printbibsection{misc}{other publications} % Print all miscellaneous entries from the bibliography
%
%\printbibsection{report}{research reports} % Print all research reports from the bibliography

%----------------------------------------------------------------------------------------

\end{document}
