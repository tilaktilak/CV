%%%%%%%%%%%%%%%%%%%%%%%%%%%%%%%%%%%%%%%%%
% Friggeri Resume/CV
% XeLaTeX Template
% Version 1.0 (5/5/13)
%
% This template has been downloaded from:
% http://www.LaTeXTemplates.com
%
% Original author:
% Adrien Friggeri (adrien@friggeri.net)
% https://github.com/afriggeri/CV
%
% License:
% CC BY-NC-SA 3.0 (http://creativecommons.org/licenses/by-nc-sa/3.0/)
%
% Important notes:
% This template needs to be compiled with XeLaTeX and the bibliography, if used,
% needs to be compiled with biber rather than bibtex.
%
%%%%%%%%%%%%%%%%%%%%%%%%%%%%%%%%%%%%%%%%%

\documentclass[a4paper]{friggeri-cv} % Add 'print' as an option into the square bracket to remove colors from this template for printing

%\addbibresource{bibliography.bib} % Specify the bibliography file to include publications

\begin{document}

\header{Pierre-Louis}{TILAK}{Embedded System Engineer} % Your name and current job title/field
%\header{Master Droit }{Juriste d'Entreprise}{Maud BERJAUD-ZARZI}{ID.jpg} % Your name and current job title/field

%----------------------------------------------------------------------------------------
%	SIDEBAR SECTION
%----------------------------------------------------------------------------------------

\begin{aside} % In the aside, each new line forces a line break
\section{Contact}
115 Rue Bonnat apt 244
31400 Toulouse
France
06 17 83 77 13
~
\href{mailto:pierre.tilak@gmail.com}{pierre.tilak@gmail.com}
%\href{etud.insa-toulouse.fr/~tilak}{etud.insa-toulouse.fr/~tilak}
\href{https://www.linkedin.com/pub/pierre-louis-tilak/96/162/a83}{linkedin://pierre-louis-tilak}
\section{Langues}
English : Fluent
TOEIC : 925 pts 
Spanish : Academic Level\bigskip\bigskip
\section{Programmation}
{\color{red} $\varheartsuit$} C/Cpp
Python, Docker, YML,
Javascript,
HTML,\bigskip\bigskip
\section{Hardware \& PC}
Embedded Linux\bigskip\bigskip
\section{Software}
Vim, Git/GitLab, Docker, Latex, MS Office\bigskip\bigskip
\section{Other}
Project Management : Agile, Scrum	
\end{aside}

%----------------------------------------------------------------------------------------
%	EDUCATION SECTION
%----------------------------------------------------------------------------------------

\section{Education}

\begin{entrylist}

%------------------------------------------------
\entry
{2010--2014}
{Master of engineering}
{Institut National des Sciences Appliquées, Toulouse}
{5ème Année - Systèmes Embarqués : 

\begin{itemize}
\item Automatique, Lois de commande, Asservissements
\item Électronique et Programmation bas niveau, Drivers, STM32
\item Systèmes Embarqués Critiques, Programmation SCADE
\item Programmation haut niveau JAVA, Android
\item OS Temps Réel, Mise en place Réseau
\end{itemize}}

%------------------------------------------------

\entry
{2013}
{Exchange Semester - Thailand}
{Study at Chulanlongkorn University - Bangkok}
{Information and Communication Engineering : 
Universal and Technical English skill refining, Embedded System, Arduino Development, Multimedia Engineering}

%------------------------------------------------

%\entry
%{2010}
%{Baccalauréat Scientifique Option Sciences de l'Ingénieur}
%{Déodat de Séverac, Toulouse}{Mention Bien}
%------------------------------------------------



\end{entrylist}




%----------------------------------------------------------------------------------------
%	WORK EXPERIENCE SECTION
%----------------------------------------------------------------------------------------

\section{Experiences}

\begin{entrylist}
%------------------------------------------------
\entry
{Now}
{Embedded System Engineer}
{Work as developper in Autopilot team}
{
\begin{itemize}
\item Progreammation C sur microcontroller : loi de pilotage de nacelle
	stabilisée, drivers, protocoles de communication
\item Développement d'outils d'ingénieurie en Python (Interfaces graphiques,
	Scripts de communication TCP, Série, de requête web, Tests Unitaires en
		BDD, ...)
\item Mise en place d'environnement de développement : Chaîne
	de cross-compilation, Intégration continue via des Runners Gitlab :
	Créer des images Docker de build, ou contenant l'environnement de test
	simulateur, dérouler les tests unitaires. Processus de génération de
	binaire automatisé via pipeline Gitlab, versionnement des binaires
	(JFrog Artifactory)
\item Programmation CPP sur linux embarqué : Configuration de capteur, Threads
	utilisant différentes IPC (Signaux Linux, SharedMemory, ZMQ, ...)
\end{itemize}}
%------------------------------------------------
\entry
{2014}
{Stage Ingénieur 3 mois}
{SCLE SFE - Toulouse}
{Développement de plan de test JTAG pour Boundary Scan (Logiciel XJTAG):\smallskip
\begin{itemize}
\item Codes test pour les SoC (composants Ethernet, SPI, I2C, Watchdog, Série)
\item Analyse, Conception et Réalisation de cartes interfaces, débogage sur banc de test.
\end{itemize}}

%------------------------------------------------

\entry
{2012}
{Rédacteur de modes opératoires}
{Polymont Sous-traitant Airbus - Toulouse}
{Mission sur démantèlement de site : Rédaction de modes opératoires sur Iron-bird Airbus A330,A320 et A380.}

%------------------------------------------------

\entry{2011}{Stage Ouvrier 1 mois}{Airbus - Toulouse}{Equipe Inspection Qualité A380 : Optimisation de la durée des inspections}


\end{entrylist}


%----------------------------------------------------------------------------------------
%	PROJETS
%----------------------------------------------------------------------------------------

\section{projets tutorés}

\begin{entrylist}
%------------------------------------------------

\entry
{6 Mois}
{AR-Drone et jeu de plateau}
{Projet 5A - INSA, Toulouse}
{AR-Drone Parrot, Cortex A8, Linux, OpenCV \smallskip
\begin{itemize}
\item Développement des primitives pour faire évoluer le drone
\item Détection de l'environnement via traitement de l'image
\item IHM interactive et contrôle du drone via superviseur
\end{itemize}
}

%------------------------------------------------

%------------------------------------------------

\entry
{6 Mois}
{Développement d'un OS de Console de Jeu}
{Projet Tutoré 4A - INSA, Toulouse}
{Carte de Développement Keil STM32\smallskip
\begin{itemize}
\item Développement des drivers (LCD,FSMC,SDIO,etc), codes d'initialisation du matériel
\item Notions de contexte applicatif et de processus
\item Chargement des informations en RAM et exécution d'un jeu depuis l'OS
\end{itemize}
}

%------------------------------------------------

%\entry
%{2013}
%{Projet Électronique}
%{Chulalongkorn, Bangkok}
%{1 mois Carte Arduino, Développement d'une boîte à musique,
%Codage d'un ensemble de primitives permettant de choisir, lancer et arrêter la lecture d'une piste audio}
%\end{entrylist}

%------------------------------------------------


\entry
{2014}
{Conception et Développement Temps Réel}
{Projet Temps Réel 4A - INSA, Toulouse}
{Conception UML, Définition des threads, des primitives de synchronisation, développement sur OS Xenomai afin de faire évoluer un robot filmé dans une arène.}

\entry
{2013}
{Assembleur ARM Cortex M3}
{Projet STM32 4A - INSA, Toulouse}
{Programmation bas niveau, données accéléromètres, afficheur à persistance rétinienne}
\end{entrylist}

%----------------------------------------------------------------------------------------
%	COMMUNICATION SKILLS SECTION
%----------------------------------------------------------------------------------------

%\section{compétences}
%
%%------------------------------------------------
%
%\emph{Systèmes Embarqués}
%
%\begin{itemize}
%\item{\textbf{Programmation }}{ Langage C, Java, Assembleur, SQL, ADA, HTML, Xjease}
%\item{\textbf{Projets }}{STM32 Cortex M3, Pics, Arduino}
%\item{\textbf{Hardware PC }}{Bonnes notions sur l'architecture, Développement bas niveau (Assembleur, Drivers,etc)}
%\item{\textbf{OS utilisés }}{Linux, Xenomai (OS Temps Réel), Windows}
%\item{\textbf{Logiciels Maîtrisés }}{Intelij-Idea, Matlab-Simulink, Keil-uVision, Cadence Orcad, Solidwork} 
%\item{\textbf{Bureautique }}{SubVersion, Tortoise, SVN, Latex, Pack Office, Gestion de base de données }
%\end{itemize}
%
%%------------------------------------------------
%
%\emph{Autres}
%\begin{itemize}
%\item{\textbf{Gestion de Projet } Mis en pratique lors de multiples projets INSA ainsi que lors de mes stages}
%\item{ \textbf{Oral } Attrait pour les présentations orales, le partage des informations}
%\item{ \textbf{Gestion \& Comptabilité } Notions pratiques acquises lors de cours et de projets de simulation d'entreprise}
%\end{itemize}
%
%%------------------------------------------------



%----------------------------------------------------------------------------------------
%	INTERESTS SECTION
%----------------------------------------------------------------------------------------

\section{Hobbies}
\textbf{Modélisme} Conception et fabrication d'avions radio-commandés
\textbf{Électronique et Informatique} Programmation de pic, Électronique, Développement Linux
\textbf{Musique} Guitariste amateur
\textbf{Sport} Paragliding, VTT, Basket-ball, Ski



%----------------------------------------------------------------------------------------
%	PUBLICATIONS SECTION
%----------------------------------------------------------------------------------------

%\section{publications}
%
%\printbibsection{article}{article in peer-reviewed journal} % Print all articles from the bibliography
%
%\printbibsection{book}{books} % Print all books from the bibliography
%
%\begin{refsection} % This is a custom heading for those references marked as "inproceedings" but not containing "keyword=france"
%\nocite{*}
%\printbibliography[sorting=chronological, type=inproceedings, title={international peer-reviewed conferences/proceedings}, notkeyword={france}, heading=subbibliography]
%\end{refsection}
%
%\begin{refsection} % This is a custom heading for those references marked as "inproceedings" and containing "keyword=france"
%\nocite{*}
%\printbibliography[sorting=chronological, type=inproceedings, title={local peer-reviewed conferences/proceedings}, keyword={france}, heading=subbibliography]
%\end{refsection}
%
%\printbibsection{misc}{other publications} % Print all miscellaneous entries from the bibliography
%
%\printbibsection{report}{research reports} % Print all research reports from the bibliography

%----------------------------------------------------------------------------------------

\end{document}
